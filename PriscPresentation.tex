%%%%%%%%%%%%%%%%%%%%%%%%%%%%%%%%%%%%%%%%%
% Beamer Presentation
% LaTeX Template
% Version 1.0 (10/11/12)
%
% This template has been downloaded from:
% http://www.LaTeXTemplates.com
%
% License:
% CC BY-NC-SA 3.0 (http://creativecommons.org/licenses/by-nc-sa/3.0/)
%
%%%%%%%%%%%%%%%%%%%%%%%%%%%%%%%%%%%%%%%%%

%----------------------------------------------------------------------------------------
%	PACKAGES AND THEMES
%----------------------------------------------------------------------------------------

\documentclass{beamer}

\mode<presentation> {

% The Beamer class comes with a number of default slide themes
% which change the colors and layouts of slides. Below this is a list
% of all the themes, uncomment each in turn to see what they look like.

%\usetheme{default}
%\usetheme{AnnArbor}
%\usetheme{Antibes}
%\usetheme{Bergen}
%\usetheme{Berkeley}
%\usetheme{Berlin}
%\usetheme{Boadilla}
%\usetheme{CambridgeUS}
%\usetheme{Copenhagen}
%\usetheme{Darmstadt}
%\usetheme{Dresden}
%\usetheme{Frankfurt}
%\usetheme{Goettingen}
%\usetheme{Hannover}
%\usetheme{Ilmenau}
%\usetheme{JuanLesPins}
%\usetheme{Luebeck}
\usetheme{Madrid}
%\usetheme{Malmoe}
%\usetheme{Marburg}
%\usetheme{Montpellier}
%\usetheme{PaloAlto}
%\usetheme{Pittsburgh}
%\usetheme{Rochester}
%\usetheme{Singapore}
%\usetheme{Szeged}
%\usetheme{Warsaw}

% As well as themes, the Beamer class has a number of color themes
% for any slide theme. Uncomment each of these in turn to see how it
% changes the colors of your current slide theme.

%\usecolortheme{albatross}
%\usecolortheme{beaver}
%\usecolortheme{beetle}
%\usecolortheme{crane}
%\usecolortheme{dolphin}
%\usecolortheme{dove}
%\usecolortheme{fly}
%\usecolortheme{lily}
%\usecolortheme{orchid}
%\usecolortheme{rose}
%\usecolortheme{seagull}
\usecolortheme{seahorse}
%\usecolortheme{whale}
%\usecolortheme{wolverine}

%\setbeamertemplate{footline} % To remove the footer line in all slides uncomment this line
%\setbeamertemplate{footline}[page number] % To replace the footer line in all slides with a simple slide count uncomment this line

%\setbeamertemplate{navigation symbols}{} % To remove the navigation symbols from the bottom of all slides uncomment this line
\setbeamertemplate{blocks}[rounded][shadow=false]

}

\usepackage{graphicx} % Allows including images
\usepackage{booktabs} % Allows the use of \toprule, \midrule and \bottomrule in tables


\usepackage{xcolor}
\usepackage{listings}

\definecolor{mGreen}{rgb}{0,0.5,0}
\definecolor{mGray}{rgb}{0.5,0.5,0.5}
\definecolor{mPurple}{rgb}{0.58,0,0.82}
\definecolor{backgroundColour}{rgb}{0.95,0.95,0.92}


\lstloadlanguages{{C}}

\lstdefinestyle{CStyle}{
   % backgroundcolor=\color{backgroundColour},   
    commentstyle=\color{mGreen},
    keywordstyle=\color{magenta},
    numberstyle=\tiny\color{mGray},
    stringstyle=\color{mPurple},
    basicstyle=\footnotesize,
    breakatwhitespace=false,         
    breaklines=true,                 
    captionpos=b,                    
    keepspaces=true,                 
    numbers=left,                    
    numbersep=5pt,                  
    showspaces=false,                
    showstringspaces=false,
    showtabs=false,                  
    tabsize=2,
    language=C,
    aboveskip=\smallskipamount,
    belowskip=\smallskipamount,
    mathescape=true,
    morekeywords = {assert},
    moredelim=**[is][\only<+>{\color{black}\lstset{style=highlight}}]{~}{~},
    literate= {|->}{{$\mapsto$}}1 {_L}{{$_L$}}1,
}

\lstdefinestyle{CStyleNoNum}{
   % backgroundcolor=\color{backgroundColour},   
    commentstyle=\color{mGreen},
    keywordstyle=\color{magenta},
    numberstyle=\tiny\color{mGray},
    stringstyle=\color{mPurple},
    basicstyle=\footnotesize,
    breakatwhitespace=false,         
    breaklines=true,                 
    captionpos=b,                    
    keepspaces=true,                 
    %numbers=left,                    
    %numbersep=5pt,                  
    showspaces=false,                
    showstringspaces=false,
    showtabs=false,                  
    tabsize=2,
    language=C,
    aboveskip=\smallskipamount,
    belowskip=\smallskipamount,
    mathescape=true,  
    morekeywords = {assert},
    moredelim=**[is][\only<+>{\color{black}\lstset{style=highlight}}]{~}{~},
    literate= {|->}{{$\mapsto$}}1 {_L}{{$_L$}}1,
}

\lstdefinestyle{highlight}{
  keywordstyle=\color{magenta},
  commentstyle=\color{mGreen},
}

\lstdefinestyle{CStyleOverlay}{
   % backgroundcolor=\color{backgroundColour}, 
    %basicstyle=\footnotesize,    
    basicstyle=\footnotesize\color{black!40},
  	keywordstyle=\color{black!40},%{red!40},
  	commentstyle=\color{black!40},%{green!40},  
    %commentstyle=\color{mGreen},
    %keywordstyle=\color{magenta},
    numberstyle=\tiny\color{mGray},
    stringstyle=\color{mPurple},
    breakatwhitespace=false,         
    breaklines=true,                 
    captionpos=b,                    
    keepspaces=true,                 
    numbers=left,                    
    numbersep=5pt,                  
    showspaces=false,                
    showstringspaces=false,
    showtabs=false,                  
    tabsize=2,
    language=C,
    aboveskip=\smallskipamount,
    belowskip=\smallskipamount,
    mathescape=true,
    morekeywords = {assert},
    moredelim=**[is][\only<+-+(1)>{\color{black}\lstset{style=highlight}}]{~}{~},
    moredelim=**[is][\only<.(1)>{\color{black}\lstset{style=highlight}}]{@@}{@@},
    moredelim=**[is][\only<6>{\color{black}\lstset{style=highlight}}]{£}{£},
    moredelim=**[is][\only<7>{\color{black}\lstset{style=highlight}}]{**}{**},
    literate= {|->}{{$\mapsto$}}1 {_L}{{$_L$}}1 {_(L-1)}{{$_{L-1}$}}3,
    escapeinside={µ}{µ},
    %moredelim=**[is][\only<+>{\color{red}}]{@@}{@@},
}
\let\origthelstnumber\thelstnumber
\makeatletter
\newcommand*\Suppressnumber{%
  \lst@AddToHook{OnNewLine}{%
    \let\thelstnumber\relax%
     \advance\c@lstnumber-\@ne\relax%
    }%
}
\newcommand*\Reactivatenumber[1]{%
  \setcounter{lstnumber}{\numexpr#1-1\relax}
  \lst@AddToHook{OnNewLine}{%
   \let\thelstnumber\origthelstnumber%
   \refstepcounter{lstnumber}
  }%
}

\makeatother
\usepackage{tikz}


\usepackage{amsmath}
\usepackage{amssymb}
\usepackage[inference]{semantic}
\usepackage{xargs}
\usepackage{etoolbox}

\makeatletter
\def\@myenvname{equation}
\newcommand{\eqIfNoEq}[1]{%
  \ifx\@currenvir\@myenvname
    #1
  \else
    \begin{equation}#1
    \end{equation}
  \fi}
\newcommand{\eqIfNoMM}[1]{%
  \ifmmode
    #1
  \else
    \begin{equation}#1
    \end{equation}
  \fi}
\makeatother

\let\oldinference\inference
%\renewcommand{\inference}[2]{\[\oldinference{#1}{#2}\]}
\renewcommandx*\inference[3][3]{
\ifstrempty{#3}{
\eqIfNoEq{\oldinference{#1}{#2}}
}{
\eqIfNoEq{\oldinference{#1}{#2\tagsc{#3}}}
}
}

\newbool{shouldUseCompile}
\setbool{shouldUseCompile}{true}
\newcommandx*\triple[7][2,3,6,7]{\{#1\}^{#2}_{#3}\ #4\ \{#5\}^{#6}_{#7}} %\ifmmode to enforce math mode, but this should work without the checks
\newcommandx*\compile[3][3]{\ifbool{shouldUseCompile}{#1 \leadsto_{#3} #2}{#1}}

\usepackage{enumitem}
\setitemize{label=\usebeamerfont*{itemize item}%
  \usebeamercolor[fg]{itemize item}
  \usebeamertemplate{itemize item}}

\newcommandx*\simrelcom[2]{
{#1} \ R\ {#2}
}
\newcommandx*\backtranslation[1]{\langle\langle #1 \rangle\rangle}



%blue boxes
\definecolor{myblue}{rgb}{.84, .84, .95}
\usepackage{empheq}
\setbeamercolor{block body}{bg=myblue}


\newlength\mytemplen
\newsavebox\mytempbox

\makeatletter
\newcommand\mybluebox{%
    \@ifnextchar[%]
       {\@mybluebox}%
       {\@mybluebox[0pt]}}

\def\@mybluebox[#1]{%
    \@ifnextchar[%]
       {\@@mybluebox[#1]}%
       {\@@mybluebox[#1][0pt]}}

\def\@@mybluebox[#1][#2]#3{
    \sbox\mytempbox{#3}%
    \mytemplen\ht\mytempbox
    \advance\mytemplen #1\relax
    \ht\mytempbox\mytemplen
    \mytemplen\dp\mytempbox
    \advance\mytemplen #2\relax
    \dp\mytempbox\mytemplen
    \colorbox{myblue}{\hspace{1em}\usebox{\mytempbox}\hspace{1em}}}

\makeatother

%----------------------------------------------------------------------------------------
%	TITLE PAGE
%----------------------------------------------------------------------------------------

\title[Linear Capability Verification]{Linear capabilities for modular fully-abstract
compilation of verified code} % The short title appears at the bottom of every slide, the full title is only on the title page

\author[Thomas Van Strydonck]{\underline{Thomas Van Strydonck} \and Dominique Devriese \and Frank Piessens} % Your name
\institute[KU Leuven] % Your institution as it will appear on the bottom of every slide, may be shorthand to save space
{
KU Leuven \\ % Your institution for the title page
\medskip
\textit{thomas.vanstrydonck@cs.kuleuven.be} % Your email address
}
\date{\today} % Date, can be changed to a custom date

\begin{document}

\begin{frame}
\titlepage % Print the title page as the first slide 
\end{frame}

%\begin{frame}
%\frametitle{Overview} % Table of contents slide, comment this block out to remove it
%\tableofcontents % Throughout your presentation, if you choose to use \section{} and \subsection{} commands, these will automatically be printed on this slide as an overview of your presentation
%\end{frame}

%----------------------------------------------------------------------------------------
%	PRESENTATION SLIDES
%----------------------------------------------------------------------------------------

%------------------------------------------------
%\section{First Section} % Sections can be created in order to organize your presentation into discrete blocks, all sections and subsections are automatically printed in the table of contents as an overview of the talk
%------------------------------------------------
\begin{frame}
\frametitle{Preserving sound modular verification}


\begin{columns}
\begin{column}{0.65\textwidth}
\begin{itemize}
\item Separation logic in verification tools
	\begin{itemize}
	\item Sound
	\item Modular
	\end{itemize}
\item \textbf{Problem}: Guarantees \emph{lost} in untrusted context %why? keeping references etc; multithreading,...
\item \textbf{Solution}: Compiler enforces separation logic contracts  %compiles vf -> uvf
\end{itemize}
\end{column}
\begin{column}{0.37\textwidth}
\def\firstcircle{(0,0) circle (2cm)}
\def\secondcircle{(0,0) circle (1.3cm)}
\colorlet{circle edge}{black!100}
\colorlet{circle area}{black!0}
\tikzset{filled/.style={fill=circle area, draw=circle edge, thick}, outline/.style={draw=circle edge, thick}}

%\setlength{\parskip}{5mm}
\begin{tikzpicture}
    \draw[outline] \firstcircle node {Verified Module};
    \draw[outline] \secondcircle node {};
    \node at (0,-1.6) (nodeA) {Context};
\end{tikzpicture}
\end{column}
\end{columns}
\end{frame}
%------------------------------------------------

\begin{frame}
\frametitle{The compiler}
\begin{columns}
\begin{column}{0.45\textwidth}
	\begin{block}{Source language}
	\begin{itemize}
	\item Regular verified C code
	\item Separation logic annotated
		\begin{itemize}
		\item e.g. VeriFast syntax for concreteness
		\end{itemize}
	\end{itemize}
	\end{block}
\end{column}
\begin{column}{0.08\textwidth}
	\begin{figure}
	\includegraphics[width=0.8\linewidth]{BlueArrow}
	\end{figure}
\end{column}
\begin{column}{0.45\textwidth}
    \begin{block}{Target language}
	\begin{itemize}
	\item Regular unverified C code
	\item Support for \emph{capabilities} \\(next slide)
		\begin{itemize}
		\item CHERI-inspired 
		\item Linear capabilities
		%\item Sealed capabilities %Q? Do we still need these? e.g. nested arrays wont contain nested capabilities; the nesting of capabilities was only a problem wiht AP's
		\end{itemize}
	\end{itemize}
	\end{block}
\end{column}
\end{columns}
%\vspace{1em}
\center No assembly hassle in C, but still unsafe (powerful attacker).

 %compiles vf -> uvf
\end{frame}

%------------------------------------------------

\begin{frame}
\frametitle{(Linear) Capabilities}

\begin{columns}
\begin{column}{0.5\textwidth}
	\textbf{Capability}: %Q? leave this out/reduce to one line? Or add figure? + source for figure?
\begin{itemize}
\item Unforgeable memory pointer
\item Grants permissions on memory region
\item Fine-grained memory protection
\item Capability machines (ex CHERI)
\end{itemize}
\end{column}
\begin{column}{0.5\textwidth}
\begin{figure}
\includegraphics[width=\linewidth]{Capability}%Q? Reference? https://www.semanticscholar.org/paper/The-CHERI-capability-model-Revisiting-RISC-in-an-a-Woodruff-Watson/0657eb7e069c2c2c7cae6636704e0f7fb3bcd9fc
\end{figure}
\end{column}
\end{columns}
\vspace{.5em}

 %compiles vf -> uvf
%\textbf{Sealed Capability}: %Q? Delete
%Unaccessible until unsealed using an appropriate seal.
%\vspace{.5em}

\textbf{Linear Capability}: %Fits very well with consuming and producing in separation logic = linear aspects of separation logic
\begin{itemize}
\item Linearity = one-use! cfr e.g. Linear Logic
\item Non-copyable
$\Rightarrow$ callers/callees cannot keep copies
\item Intuitive: separation logic is linear
\end{itemize}
\end{frame}

%------------------------------------------------
\begin{frame}
\frametitle{Relation to \emph{full abstraction}}
 \setbeamercovered{transparent}
\begin{block}{Full abstraction}<1>
= reflection and preservation of contextual equivalence \\
\vspace{-1em}
\begin{align*}&s\simeq_{ctx}s' \Leftrightarrow [[s]]\simeq_{ctx}[[s']]\\
&\text{where } x\simeq_{ctx}x'\equiv \forall C: C[x]\Downarrow\ \Leftrightarrow C[x']\Downarrow
\end{align*}
$\supseteq$ preservation of integrity and confidentiality \\
\end{block}
\only<2>{
\begin{block}{Importance} %Q? or is this too trivial? It was a confusing bit for me at the start
Fully abstract compiler $\Rightarrow$ compiled code upholds contracts\\<2>
\end{block}
\textbf{Related work (Agten et al.)} %Q? maybe leave out? This is super relevant though, and underscores the advantage of fine-grainedness of capabilities wrt PMAs, but not everyone might know this
\begin{itemize}
\item Different hardware primitives\\%PMA's instead of capabilities\\
	$\Rightarrow$ Less fine-grained
\item Integrity, \emph{not} confidentiality

\end{itemize}
} 
 %compiles vf -> uvf
\end{frame}

%------------------------------------------------
\begin{frame}[fragile]
\frametitle{Example program} %Q? zie code: afwijken van C - syntax; eerder in slides ook C-like schrijven? (anders ga je te veel moeten focussen op het low-level verschil tussen een pointer en een reference denk ik)
Illustrates approach\\
Based on \emph{separation logic derivation} (next slide)
\begin{columns}
\begin{column}{0.015\textwidth}
\end{column}
\begin{column}{0.55\textwidth}
\begin{figure}[h]
  \centering
\begin{lstlisting}[style=CStyle, captionpos = t]
void array_map(int n[], int *data, int L)
//@pre n |-> [_]_L * data |-> _;
//@post n |-> [_]_L * data |-> _;
{
	if (L == 0) { 
		skip;
	} else {
		//@split n[0];
		int newVal = p(n[0], data);
		n[0] = newVal;
		array_map(n+1, data, L-1);
		//@join n (n+1);
	}
	return; 
}
\end{lstlisting}
\end{figure}
\end{column}
\begin{column}{0.43\textwidth}
\begin{lstlisting}[style=CStyle, captionpos = t]
int p(int x, int *data)
//@pre data $\textcolor{mGreen}{\mapsto}$ _; 
//@post data $\textcolor{mGreen}{\mapsto}$ _;
{$\ldots$}
\end{lstlisting}
\begin{block}{Elements}
\begin{itemize}
\item $\textcolor{mGreen}{\ast}$, $\textcolor{mGreen}{\mapsto}$ 
\item @pre/post: contract
\item array chunk notation: [$\cdot$]
\item @split/join: manipulate array chunks
\end{itemize}
\end{block}
\end{column}
\begin{column}{0.005\textwidth}
\end{column}
\end{columns}
\end{frame}
%------------------------------------------------
\begin{frame}
\frametitle{Separation logic derivation}
= tree-shaped proof of function contract\\
%= tree of Hoare triples related by separation logic axioms (e.g. \textsc{If}, \textsc{F-App}, $\ldots$)\\%Q? maybe show one step
\vspace{.5em}
\textbf{Root} = the Hoare triple: $\{pre\}\ BODY\ \{post\}$\\
\vspace{.5em}
Used as input $\Rightarrow$ \emph{separation-logic-proof-directed compilation}
%In-line representation as \emph{symbolic execution}
\end{frame}
%------------------------------------------------
\iffalse
\begin{frame}[fragile]
\frametitle{Example program: if derivation}
\begin{columns}
\begin{column}{0.03\textwidth}
\end{column}
\begin{column}{0.97\textwidth}
\begin{figure}[h]
  \centering
\begin{lstlisting}[style=CStyleOverlay, captionpos = t]
void array_map(int n[], int *data, int L)
@@//@pre n |-> [_]_L * data |-> _;
@@£//@post n |-> [_]_L * data |-> _;
£{
	~//{c1: n |-> [_]_L * c2: data |-> _}
	~@@if (L == 0) {@@
		~//{c1: n |-> [_]_L * c2: data |-> _ * L == 0}
		~@@skip;@@
		~//{c1: n |-> [_]_L * c2: data |-> _ * L == 0}
	~} else {
		(...)
	}
	@@return;@@
	~//{c1: n |-> [_]_L * c2: data |-> _ * L == 0 * result == ()}
	~~//{c1: n |-> [_]_L * c2: data |-> _ }
~}
\end{lstlisting}
\end{figure}
\end{column}
\end{columns}
\end{frame}
\fi
%------------------------------------------------
\begin{frame}[fragile]
\frametitle{Example program: else derivation}
\vspace{-2em}
\begin{columns}
\begin{column}{0.03\textwidth}
\end{column}
\begin{column}{0.97\textwidth}
\begin{figure}[h]
  \centering
\begin{lstlisting}[style=CStyleOverlay, captionpos = t]
void array_map(int n[], int *data, int L)
@@//@pre n |-> [_]_L * data |-> _;
@@//@post n |-> [_]_L * data |-> _;
{
	~//{c1: n |-> [_]_L * c2: data |-> _}
	~@@if (L == 0) {@@
		(...)
	} else {
		~//{c1: n |-> [_]_L * c2: data |-> _ * L != 0}
		~~//{c1: n |-> [d,_]_L * c2: data |-> _}
		~@@//@split n[0];
		@@~//{c1: n |-> [d] * c3: n+1 |-> [_]_(L-1) * c2: data |-> _}
		~@@int newVal = p(n[0], data);@@
		~//{c1: n |-> [d] * c3: n+1 |-> [_]_(L-1) * c2: data |-> _ * newVal = _}
		~@@n[0] = newVal;@@
		~//{c1: n |-> [newVal] * c3: n+1 |-> [_]_(L-1) * c2: data |-> _}
		~(...)		
\end{lstlisting}
\end{figure}
\end{column}
\end{columns}
\end{frame}
%------------------------------------------------
\begin{frame}[fragile]
\frametitle{Example program: else derivation}
\vspace{-2em}
\begin{columns}
\begin{column}{0.03\textwidth}
\end{column}
\begin{column}{0.97\textwidth}
\begin{figure}[h]
  \centering
\begin{lstlisting}[style=CStyleOverlay, captionpos = t]
void array_map(int n[], int *data, int L)
//@pre n |-> [_]_L * data |-> _;
**//@post n |-> [_]_L * data |-> _;
**{
		(...)	µ\Suppressnumberµµ\Reactivatenumber{16}µ
		~//{c1: n |-> [newVal] * c3: n+1 |-> [_]_(L-1) * c2: data |-> _}
		~@@array_map(n+1, data, L-1);
		@@~//{c1: n |-> [newVal] * c3: n+1 |-> [_]_(L-1) * c2: data |-> _}
		~@@//@join n (n+1);
		@@~//{c1: n |-> [newVal,_]_L * c2: data |-> _}
		~~//{c1: n |-> [_]_L * c2: data |-> _}
	~}
	@@return;@@
	~//{c1: n |-> [_]_L * c2: data |-> _ * result == ()}
	~~//{c1: n |-> [_]_L * c2: data |-> _ }
~}
\end{lstlisting}
\end{figure}
\end{column}
\end{columns}
\end{frame}
%------------------------------------------------
\begin{frame}[fragile]
\frametitle{Translation: Intuition}
\begin{columns}
\begin{column}{0.6\textwidth}
\emph{Separation-logic-proof-directed}
\begin{itemize}
\item Chunks become linear capabilities
	\begin{itemize}
	\item Contain all permissions
	%\item Passed to/from functions
	\item This is why we \textbf{name} heap chunks!
	\end{itemize}
\item Original pointers become addresses
	\begin{itemize}
	\item Regular ints
	\item Lose all permission
	\item Kept for address operations
	\end{itemize} 
%\item Enforcing function contracts for
%	\begin{itemize}
%	\item Incalls
%	\item Outcalls
%	\end{itemize}
%	$\Rightarrow$ Stubs
\end{itemize}
\end{column}
\begin{column}{0.4\textwidth}

\begin{center}
\begin{tabular}{c}
\begin{lstlisting}[style=CStyleNoNum, captionpos = t]
//{c1: n |-> [_]_L}
n: int*
\end{lstlisting}
\end{tabular}
\end{center}

\vspace{-.5em}
\begin{figure}[h]
\centering
\includegraphics[width=0.20\linewidth]{BlueArrowVertical}
\end{figure}
\vspace{-1em}

\begin{center}
\begin{tabular}{c}
\begin{lstlisting}[style=CStyleNoNum, captionpos = t]
c1: int* (linear)
 n: int
\end{lstlisting}
\end{tabular}
\end{center}

\end{column}
\end{columns}
\end{frame}

%------------------------------------------------
%Translation: template
\begin{frame}[fragile]
\frametitle{Translation: Split/Join}
Performed by target language built-in functions \emph{split} and \emph{join}.\\
Split creates 2 linear array capabilities out of one. Join does the opposite.

\vspace{-1em}

\begin{columns}
\begin{column}{0.15\textwidth}
\end{column}
\begin{column}{0.90\textwidth}
\begin{figure}[h]
  \centering
\begin{lstlisting}[style=CStyleOverlay, captionpos = t, firstnumber = 9]
//{c1: n |-> [_]_L * c2: data |-> _ * L != 0}
@@//{c1: n |-> [d,_]_L * c2: data |-> _}
//@split n[0];
//{c1: n |-> [d] * c3: n+1 |-> [_]_(L-1) * c2: data |-> _}
@@int newVal = p(n[0], data);
\end{lstlisting}
\end{figure}
\end{column}
\end{columns}

\vspace{-.5em}
\begin{figure}[h]
\includegraphics[width=0.07\linewidth]{BlueArrowVertical}
\end{figure}
\vspace{-2em}

\begin{center}
\begin{tabular}{c}
\begin{lstlisting}[style=CStyleNoNum, captionpos = t]
{c1,c3} = split(c1,0);
\end{lstlisting}
\end{tabular}
\end{center}
\end{frame}
%------------------------------------------------
%Translation: template
\begin{frame}[fragile]
\frametitle{Translation: Array operations}
Source language array mutation/lookup\\
$\Rightarrow$ Target language mutation/lookup on the corresponding capability
 \vspace{-1em}

\begin{columns}
\begin{column}{0.1\textwidth}
\end{column}
\begin{column}{0.90\textwidth}
\begin{figure}[h]
  \centering
\begin{lstlisting}[style=CStyleOverlay, captionpos = t, firstnumber = 13]
int newVal  = p(n[0], data);
@@//{c1: n |-> [d] * c3: n+1 |-> [_]_(L-1) * c2: data |-> _ * newVal = _}
n[0] =  newVal;
//{c1: n |-> [newVal] * c3: n+1 |-> [_]_(L-1) * c2: data |-> _}
@@array_map(n+1, data);
\end{lstlisting}
\end{figure}
\end{column}
\end{columns}

\vspace{-.5em}
\begin{figure}[h]
\includegraphics[width=0.07\linewidth]{BlueArrowVertical}
\end{figure}
\vspace{-2em}

\begin{center}
\begin{tabular}{c}
\begin{lstlisting}[style=CStyleNoNum, captionpos = t]
c1[0] = newVal;
\end{lstlisting}
\end{tabular}
\end{center}

\end{frame}
%------------------------------------------------
%Translation: template
\begin{frame}[fragile]
\frametitle{Translation: Function call}
\begin{columns}
\begin{column}{0.55\textwidth}
Add arguments/return values to calls.\\
Corresponding to heap chunks.
\end{column}
\begin{column}{0.42\textwidth}
\begin{lstlisting}[style=CStyle, captionpos = t]
int p(int x,int *data)
//@pre data |-> _; 
//@post data |-> _;
{$\ldots$}
\end{lstlisting}
\end{column}
\end{columns}
\vspace{-1em}

\begin{columns}
\begin{column}{0.15\textwidth}
\end{column}
\begin{column}{0.90\textwidth}
\begin{figure}[h]
  \centering
\begin{lstlisting}[style=CStyleOverlay, captionpos = t, firstnumber = 11]
//@split n[0];
@@//{c1: n |-> [d] * c3: n+1 |-> [_]_(L-1) * c2: data |-> _}
int newVal = p(n[0], data);
//{c1: n |-> [d] * c3: n+1 |-> [_]_(L-1) * c2: data |-> _ * newVal = _}
@@n[0] = newVal;
\end{lstlisting}
\end{figure}
\end{column}
\end{columns}

\vspace{-.5em}
\begin{figure}[h]
\includegraphics[width=0.07\linewidth]{BlueArrowVertical}
\end{figure}
\vspace{-2em}

\begin{center}
\begin{tabular}{c}
\begin{lstlisting}[style=CStyleNoNum, captionpos = t]
{int,int*} p(int x, int data, int* data_cap){...}
           {newVal,c2} = p(c1[0],data,c2);
\end{lstlisting}
\end{tabular}
\end{center}

\end{frame}
%------------------------------------------------
%Translation: template
\begin{frame}[fragile]
\frametitle{Translation: In-/Outcall}

\begin{columns}
\begin{column}{0.55\textwidth}
\begin{itemize}
\item In/outcall stub:\\transitions to/from verified module
\item Functions:
	\begin{itemize}
	\item Check linearity of capabilities
	\item Check any conditions in the contract
	\end{itemize}
\end{itemize}
\end{column}
\begin{column}{0.42\textwidth}
\begin{lstlisting}[style=CStyle, captionpos = t]
int p(int x,int *data)
//@pre data |-> _; 
//@post data |-> _;
{$\ldots$}
\end{lstlisting}
\end{column}
\end{columns}
\vspace{-1em}

\begin{columns}
\begin{column}{0.15\textwidth}
\end{column}
\begin{column}{0.90\textwidth}
\begin{figure}[h]
  \centering
\begin{lstlisting}[style=CStyleNoNum, captionpos = t]
{int,int*} p$_{stub}$(int x, int data, int* data_cap){
	intptr_t cap_address = (intptr_t) data_cap;
	
	{int result, int* data_cap} = p(x,data,data_cap);
	
	assert(cap_address == (intptr_t) data_cap);
	assert(is_linear(data_cap));
	
	return {result, data_cap};
}
\end{lstlisting}
\end{figure}
\end{column}
\end{columns}
\end{frame}
%------------------------------------------------
\begin{frame}
\frametitle{Proof: Outline}
\begin{block}{Full abstraction}
\vspace{-1em}
\begin{align*}
&\compile{\vdash s}{t}\ \land \compile{\vdash s'}{t'} \Rightarrow
 (s\simeq_{ctx}s'\ {\color{red}\Leftrightarrow}\ t\simeq_{ctx}t')
\end{align*}
\vspace{-1.5em}
\only<1>{
\begin{itemize}[leftmargin=3cm]
\item   $\vdash s$: specific proof of $s$
\item $\compile{}{}$: compiles to
\item $x\simeq_{ctx}x'\equiv \forall C: C[x]\Downarrow\ \Leftrightarrow C[x']\Downarrow$\\
Requires operational semantics!
\end{itemize}}
\end{block}
\vspace{-1em}
\only<2>{
\begin{columns}
\begin{column}{0.45\textwidth}
	\begin{figure}
 	\includegraphics[width=0.2\linewidth,angle=-45]{BlueArrowVertical}
 	\end{figure}
 	\vspace{-1.5em}
	\begin{block}{Correctness  ${\color{red}\Leftarrow}$ }
 Reflection of $\simeq_{ctx}$
	\end{block}
\end{column}
%\begin{column}{0.08\textwidth}
%	\begin{figure}
%	\includegraphics[width=0.8\linewidth]{BlueArrow}
%	\end{figure}
%\end{column}
\begin{column}{0.45\textwidth}
	\begin{figure} 	
 	\includegraphics[width=0.2\linewidth,angle=45]{BlueArrowVertical}
 	 \end{figure}
 	 \vspace{-1.5em}
    \begin{block}{Security ${\color{red}\Rightarrow}$}
Preservation of $\simeq_{ctx}$
	\end{block}
\end{column}
\end{columns}
%Give some kind of intuition here (while speaking)!
}
\end{frame}
%------------------------------------------------
\begin{frame}
\frametitle{Proof: Correctness ${\color{red}\Leftarrow}$}
\begin{block}{Approach}
\begin{equation*}
\oldinference{
t\simeq_{ctx}t'\\
\compile{\vdash s}{t} & \compile{\vdash s'}{t'}\\
}{
s\simeq_{ctx}s'}
\end{equation*}
\center Techniques: compilation $[[\cdot]]$ + simulation relation $R$
\end{block}
\end{frame}

\begin{frame}
\frametitle{Proof: Correctness}
\vspace{-1em}
\begin{columns}
\begin{column}{0.65\textwidth}
\end{column}
\begin{column}{0.35\textwidth}
\begin{empheq}[box={\mybluebox[2pt][2pt]}]{flalign*}
\oldinference{
t\simeq_{ctx}t'\\
\compile{\vdash s}{t} & \compile{\vdash s'}{t'}\\
}{
s\simeq_{ctx}s'}
\end{empheq}
\end{column}

\end{columns}

\vspace{-1em}
%first & is for left alignment,third for middle and fifth for right
\begin{alignat*}{3}
&&{\color{red}{s}}\ &{\color{red}{\simeq^?_{ctx}s'}}&&\\
&&&\Updownarrow& \text{definition } \simeq_{ctx}&\\
 \forall C\ldotp\quad C[s&]\Downarrow\ &&{\color{red}{\Rightarrow^?}}  & C[s'&]\Downarrow \\
&\Downarrow coherence &&&&\Uparrow coherence\\
\vdash C[s&]\Downarrow\ && & \vdash C[s'&]\Downarrow \\
&\Downarrow {\color{mGreen}{*}} &&&&\Uparrow {\color{mGreen}{*}}\\
[[C]][t&]\Downarrow\ &&\Rightarrow & [[C]][t'&]\Downarrow \\
&&&\Updownarrow& \text{definition } \simeq_{ctx}&\\
&&{\color{mGreen}{t}}\ &{\color{mGreen}{\simeq_{ctx}t'}}&&\\
\end{alignat*}
\vspace{-3em}
\begin{equation*}
{\color{mGreen}{*}}\ \textbf{Lemma's: }\simrelcom{\vdash C[s]}{[[C]]{\color{mGreen}{[}}[[s]]{\color{mGreen}{]}}},\ \simrelcom{\vdash x}{y} \Rightarrow\ \vdash x \Downarrow\ \Leftrightarrow y \Downarrow 
\end{equation*}
\end{frame}
%------------------------------------------------
\begin{frame}
\frametitle{Proof: Security ${\color{red}\Rightarrow}$}
\begin{block}{Approach}
\begin{equation*}
\oldinference{
s\simeq_{ctx}s'\\ 
\compile{\vdash s}{t} & \compile{\vdash s'}{t'}\\
}{
t\simeq_{ctx}t'}
\end{equation*}
\center Techniques: back-translation $\backtranslation{\cdot}$ + simulation relation $R$
\end{block}
\end{frame}
%------------------------------------------------
\begin{frame}
\frametitle{Proof: Security}
\vspace{-1em}
\begin{columns}
\begin{column}{0.65\textwidth}
\end{column}
\begin{column}{0.35\textwidth}
\begin{empheq}[box={\mybluebox[2pt][2pt]}]{flalign*}
\oldinference{
s\simeq_{ctx}s'\\ 
\compile{\vdash s}{t} & \compile{\vdash s'}{t'}\\
}{
t\simeq_{ctx}t'}
\end{empheq}
\end{column}

\end{columns}

\vspace{-1em}
%first & is for left alignment,third for middle and fifth for right
\begin{alignat*}{3}
&&{\color{mGreen}{s}}\ &{\color{mGreen}{\simeq_{ctx}s'}}&&\\
&&&\Updownarrow& \text{definition } \simeq_{ctx}&\\
  \backtranslation{C}[s&]\Downarrow\ && \Rightarrow  & \backtranslation{C}[s'&]\Downarrow \\
&\Uparrow coherence &&&&\Downarrow coherence\\
\vdash \backtranslation{C}[s&]\Downarrow\ && & \vdash \backtranslation{C}[s'&]\Downarrow \\
&\Uparrow {\color{mGreen}{*}} &&&&\Downarrow {\color{mGreen}{*}}\\
\forall C\ldotp\quad C [t&]\Downarrow\ &&{\color{red}{\Rightarrow^?}} & \vdash C[t'&]\Downarrow \\
&&&\Updownarrow& \text{definition } \simeq_{ctx}&\\
&&{\color{red}{t}}\ &{\color{red}{\simeq^?_{ctx}t'}}&&\\
\end{alignat*}
\vspace{-3em}
\begin{equation*}
{\color{mGreen}{*}}\ \textbf{Lemma's: }\simrelcom{(\vdash \backtranslation{C}[s])}{C{\color{mGreen}{[}}[[s]]{\color{mGreen}{]}}},\ \simrelcom{\vdash x}{y} \Rightarrow\ \vdash x \Downarrow\ \Leftrightarrow y \Downarrow 
\end{equation*}
\end{frame}

%------------------------------------------------

\begin{frame}[fragile]
\frametitle{Proof: Security - back-translation example}
\textbf{Intuition}:\\
\begin{itemize}
\item Construct minimal contract for context functions\\
\item Insert assertions where necessary\\
\item \textbf{Goal}: prove that $\vdash \backtranslation{C}[s] \Downarrow\ \Leftrightarrow C[t] \Downarrow $

\end{itemize}

\textbf{Example}:\\
\begin{columns}
\begin{column}{0.4\textwidth}

\begin{figure}[h]
  \centering
\begin{lstlisting}[style=CStyleNoNum, captionpos = t,title = Target]
int f(int* a, int b){
	free(a);
	b = 5; 
	return b;
}
\end{lstlisting}
\end{figure}
	
\end{column}

\begin{column}{0.08\textwidth}

	{\large$\backtranslation{\cdot}$}\\\vspace{-1em}
	\begin{figure}
	\includegraphics[width=0.8\linewidth]{BlueArrow}
	\end{figure}

\end{column}

\begin{column}{0.56\textwidth}

\begin{figure}[h]
  \centering
\begin{lstlisting}[style=CStyleNoNum, captionpos = t,title = Source]
int f(int* a, int b)
//@pre (a${\color{mGreen}{_{chunk}}}$: a |-> [_]${\color{mGreen}{_L) \lor a = null}}$;
//@post true;
{
	(...)
}
\end{lstlisting}
\end{figure}   

\end{column}
\end{columns}

\end{frame}
%------------------------------------------------
\begin{frame}[fragile]
\frametitle{Proof: Security - back-translation example}
\textbf{Intuition}:\\
\begin{itemize}
\item Construct minimal contract for context functions\\
\item Insert assertions where necessary\\
\end{itemize}

\textbf{Example}:\\
\begin{columns}
\begin{column}{0.4\textwidth}

\begin{figure}[h]
  \centering
\begin{lstlisting}[style=CStyleNoNum, captionpos = t,title = Target]
int f(int* a, int b){
	free(a);
	(...)
}
\end{lstlisting}
\end{figure}
	
\end{column}

\begin{column}{0.08\textwidth}

	{\large$\backtranslation{\cdot}$}\\\vspace{-1em}
	\begin{figure}
	\includegraphics[width=0.8\linewidth]{BlueArrow}
	\end{figure}

\end{column}

\begin{column}{0.56\textwidth}

\begin{figure}[h]
  \centering
\begin{lstlisting}[style=CStyleNoNum, captionpos = t,title = Source]
int f(int* a, int b)
//@pre (a${\color{mGreen}{_{chunk}}}$: a |-> [_]${\color{mGreen}{_L) \lor a = null}}$;
//@post true;
{
	//{(a${\color{mGreen}{_{chunk}}}$: a |-> [_]${\color{mGreen}{_L) \lor a = null}}$}
	assert(a != null);
	//{a${\color{mGreen}{_{chunk}}}$: a |-> [_]${\color{mGreen}{_L}}$}
	free(a);
	//{}
	(...)
}
\end{lstlisting}
\end{figure}   

\end{column}
\end{columns}

\end{frame}
%------------------------------------------------

\begin{frame}[fragile]
\frametitle{Proof: Security - back-translation example}
\textbf{Intuition}:\\
\begin{itemize}
\item Construct minimal contract for context functions\\
\item Insert assertions where necessary\\
\end{itemize}

\textbf{Example}:\\
\begin{columns}
\begin{column}{0.4\textwidth}

\begin{figure}[h]
  \centering
\begin{lstlisting}[style=CStyleNoNum, captionpos = t,title = Target]
int f(int* a, int b){
	(...)
	b = 5; 
	return b;
}
\end{lstlisting}
\end{figure}
	
\end{column}

\begin{column}{0.08\textwidth}

	{\large$\backtranslation{\cdot}$}\\\vspace{-1em}
	\begin{figure}
	\includegraphics[width=0.8\linewidth]{BlueArrow}
	\end{figure}

\end{column}

\begin{column}{0.56\textwidth}

\begin{figure}[h]
  \centering
\begin{lstlisting}[style=CStyleNoNum, captionpos = t,title = Source]
int f(int* a, int b)
//@pre (a${\color{mGreen}{_{chunk}}}$: a |-> [_]${\color{mGreen}{_L) \lor a = null}}$;
//@post true;
{
	(...)
	//{}
	b = 5;
	//{b = 5} 
	return b;
	//{result = 5 * b = 5}
	//{}
}
\end{lstlisting}
\end{figure}   

\end{column}
\end{columns}

\end{frame}
%------------------------------------------------
\begin{frame}
\frametitle{Conclusion and future work}
\begin{itemize}
\item Compiler from verified C to unverified C with (linear) capabilities 
\item \textbf{Claim}: Full Abstraction\\
\qquad$\Rightarrow$ Gave some proof intuition
\item \textbf{State}: \\
\qquad Correctness: $\sim$ proven\\
 \qquad Security: currently constructing back-translation\\
\end{itemize}
\end{frame}


\end{document}
